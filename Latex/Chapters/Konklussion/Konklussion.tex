%!TEX root = ../../Main.tex
\graphicspath{{Chapters/Konklussion/}}
%-------------------------------------------------------------------------------

\section{Diskussion}
Igennem projektet har vi stødt på flere forskellige problemer. Et af de meget tidskrævende problemer var identifikation af overføringsfunktionen for vores bil. Vores bil accelerede op til max hastighed utrolig hurtigt, hvilket resulteret i at det transiente område af \autopageref{fig:Motor_deg_graf} blev meget smalt. Det betød at vi kun havde nogle få samples til at bestemme overfæringsfunktionen ud fra. Desuden var der også problemer den første motor, som havde en masse slør der resulterede i at den "hoppede", hvilket teknisk set betød at vi havde en ekstra pol. Vi valgte dog at prøve en anden motor, som viste sig næsten at løse problemt. 
EV3 hardwaren har også voldt os problemer. Det har især været at få forbindelse til EV3 over wifi. Det viste sig at være et problem på en af telefonerne, så det blev løst ved at skifte telefon.
Grundet de mange problemer har vi ikke nået så langt som vi havde forventet, dog er der blevet udarbejdet et godt fungerende kontrolsystem.

\section{Konklussion}

Vi har i dette projekt, fået stillet nogle overordnede krav, til hvad vores regulerings system skulle indeholde. Vi har fra staten ville gribe dette projekt simpelt an, da vi hellere ville stå i en situation, hvor projektet skulle udvides, frem for indskrænkes. Efter forskellige test af sensorer og aktuatorer, endte vores system med at bestå af en motor og en encoder, som skulle være i stand til at køre en specifik afstand, upåvirket af vægt eller andre forstyrrelser.\\
Igennem arbejdet med dette system, kom vi bl.a. omkring identifikation af det faktiske system, placering af poler for at opnå ønsket overshoot og settling time, samt design af observer både i kontinuere og diskret domæne. \\
Selvom vi i gruppen løb ind i en del problemer, bl.a. at oprette forbindelse til HW og fjerne steady state fejlen i det diskrete domæne, fik vi alligevel implementeret en velfungerende observer i det diskrete domæne på vores Lego bil, som fungerede efter hensigten. Grundet tidsmangel og mange problemer undervejs, nåede gruppen aldrig at implementere adaptiv regulering af nogen form på systemet. \\
Alt i alt er gruppen meget tilfreds med projektforløbet, da det har været utrolig lærerigt. Det har givet en helt anden indsigt i arbejdet med regulering, end opgaver i timen her givet. 




