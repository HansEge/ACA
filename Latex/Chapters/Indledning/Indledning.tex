%!TEX root = ../../Main.tex
\graphicspath{{Chapters/Indledning/}}
%-------------------------------------------------------------------------------

\section{Indledning}
Til dette eksamensprojekt var der opstillet nogle krav til selve systemet, som vi skulle arbejde på. Systemet skulle være et dynamisk system, med mindst én pol. Systemet skulle være stabilt, og tilgå mindst én aktuator og én sensor via EV3 hardware, hvor igennem regulatoren også skulle implementeres.\\
Vores første ide til projektet var at lave en ”adaptiv følgebil”. Tanken var, at vores Lego bil kunne holde en bestemt afstand til et objekt i bevægelse, hvor svingende hastighed, skiftende hældning på underlaget, samt ændring af vægt monteret på vores Lego bil, ikke ville påvirke dens adfærd.
Efter vi havde lavet test med udstyret til rådighed, besluttede vi os for at gå i en anden retning. Vi fik nemlig mere præcise målinger fra encoderen monteret inde i motoren, og ønskede derfor at gå videre den som vores sensor. Det betød dog også, at vores Lego bil ikke længere ville være i stand til at følge et objekt i bevægelse, da den ikke længere havde ”øjne” i form af en afstandssensor. \\
Ideen vi gik videre med, blev derfor en Lego bil der vil være i stand til at køre en specifik afstand, upåvirket af f.eks. vægt monteret på Lego bilen eller underlagets hældning. Derfor endte vores system i sidste ende med at bestå af en motor, med en indlagt encoder som sensor. 

	
\newpage

